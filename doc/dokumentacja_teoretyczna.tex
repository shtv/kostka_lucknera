\documentclass[12pt]{article}

\usepackage{polski}
\usepackage[utf8]{inputenc}
\usepackage[pdfborder={0 0 0}]{hyperref}
\usepackage{hyphenat}

\hypersetup{
	pdfinfo={
		Title = {Dokumentacja teoretyczna},
		Subject = {Dokumentacja teoretyczna},
		Author = {Paweł Baran, Andrzej Legucki},
		Keywords = {kostka, dokumentacja, teoretyczna}
	}
}

\title{
%\Huge{
%Teoria Algorytmów i Obliczeń
%}
\Huge{
%Laboratoria
Zima 2010 - kostka
} \\[0.5em]
\LARGE{
Dokumentacja teoretyczna
}
}

\author{
	Paweł Baran \and Andrzej Legucki
}

\begin{document}

\maketitle

\newpage

\tableofcontents

\newpage
\section{Wstęp}
Niniejsza dokumentacja powstała w ramach pierwszego etapu laboratoriów
z~przedmiotu Teoria Algorytmów i Obliczeń. W niniejszym dokumencie
przedstawiamy opis algorytmu dokładnego oraz opisy dwu algorytmów
aproksymujących do zadania. Problemem jest ułożenie największego
pełnego prostopadłościanu z klocków. Pojedynczym zadaniem problemu jest
ułożenie pełnego prostopadłościanu z zadanego zestawu klocków.
Rozwiązaniem zadania jest sekwencja ruchów klocków należących do zestawu
z zadania.

Oprócz algorytmów przedstawione zostały opisy struktur danych potrzebnych
do funkcjonowania podanych algorytmów. Opis struktur danych ułatwi ich
obsługę w zakresie zapisu zestawu klocków i zapisu sekwencji ruchów.
Pierwszy zapis służy do zapisania danych wejściowych, drugi -- do zapisania
wyniku danego zadania sporządzonego przez zadany algorytm.

\section{Algorytm dokładny}

\section{Algorytmy aproksymacyjne}

\subsection{Algorytm klonowania}

\subsection{Algorytm Andrzeja}

\section{Struktury danych}

\subsection{Zestaw klocków}
Zestaw klocków jest daną wejściową naszego programu. Niniejszy rozdział
dzieli się na dwie części.

\subsubsection{Format XML}
Zestaw klocków jest zapisywany w dokumencie XML, którego korzeniem jest
$bricks$. W korzeniu występują jedynie elementy $brick$. Każdy element
$brick$ jest opisem dokładnie jednego klocka. Każdy klocek należący do
zestawu i tylko taki klocek jest opisany za pomocą elementu $brick$.
Element $brick$ zawiera elementy $cube$. Każdy element $cube$ ma podane
wartości trzech atrybutów: $x$, $y$ i $z$.

Podobnie każdy klocek składa się
z sześcianów, które są niepodzielne. Sześcianem jest najmniejszy możliwy
klocek o długości krawędzi równej $1$. Z tego wynika, że krawędzie klocków
mają wartości całkowite równe co najmniej $1$. Każdy sześcian danego
klocka możemy zidentyfikować za pomocą trzech współrzędnych.

Klocek jest
umieszczany w układzie współrzędnych. Dzięki takiemu zabiegowi każde dwa
różne sześciany danego klocka mają różne wektory pomieszczenia. Wektorem
pomieszczenia sześcianu nazywamy wektor złożony z trzech współrzędnych
określających położenie sześcianu względem początku układu. Każdy
sześcian klocka i tylko taki sześcian jest opisany przez element $cube$
zawarty w $brick$ danego klocka.

Sposób na zapis danego klocka, a dokładnie -- jego sześcianów, jest ściśle
określony. Zapis klocka musi spełniać następujące warunki:
\begin{enumerate}
	\item każdy sześcian klocka ma współrzędne nieujemne,
	\item nie istnieje inny zapis klocka spełniającego warunek 1, którego
		suma współrzędnych sześcianów jest mniejsza od sumy współrzędnych 
		sześcianów danego,
	\item do zrobienia\ldots
\end{enumerate}
Ten zapis \textbf{gwarantuje}, że sprawdzenie, czy dwa klocki mają
identyczny kształt, wymagać będzie jedynie sprawdzenia równości zbiorów
sześcianów danych klocków. Przez identyczny kształt rozumiemy doprowadzenie
operacjami rotacji i przemieszczania do pokrycia sześcianów pierwszego
sześcianami drugiego i odwrotnie.

\subsection{Sekwencja ruchów}
\end{document}
