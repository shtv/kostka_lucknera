\documentclass[12pt]{article}

\usepackage{polski}
\usepackage[utf8]{inputenc}
\usepackage[pdfborder={0 0 0}]{hyperref}
\usepackage{hyphenat}
\usepackage{indentfirst}

\hypersetup{
	pdfinfo={
		Title = {Dokumentacja teoretyczna},
		Subject = {Dokumentacja teoretyczna},
		Author = {Paweł Baran, Andrzej Legucki},
		Keywords = {kostka, dokumentacja, teoretyczna}
	}
}

\title{
%\Huge{
%Teoria Algorytmów i Obliczeń
%}
\Huge{
%Laboratoria
Zima 2010 - kostka
} \\[0.5em]
\LARGE{
Dokumentacja teoretyczna
}
}

\author{
	Paweł Baran \and Andrzej Legucki
}

\begin{document}

\maketitle

\newpage

\tableofcontents

\newpage
\section{Wstęp}
Niniejsza dokumentacja powstała w ramach pierwszego etapu laboratoriów
z~przedmiotu Teoria Algorytmów i Obliczeń. W niniejszym dokumencie
przedstawiamy opis algorytmu dokładnego oraz opisy dwu algorytmów
aproksymujących do zadania. Problemem jest ułożenie największego
pełnego prostopadłościanu z klocków, którego najdłuższa krawędź jest niedłuższa niż
długość najdłuższego klocka. Pojedynczym zadaniem problemu jest
ułożenie pełnego prostopadłościanu z zadanego zestawu klocków.
Rozwiązaniem zadania jest sekwencja ruchów klocków należących do zestawu
z zadania.

Oprócz algorytmów przedstawione zostały opisy struktur danych potrzebnych
do funkcjonowania podanych algorytmów. Opis struktur danych ułatwi ich
obsługę w zakresie zapisu zestawu klocków i zapisu sekwencji ruchów.
Pierwszy zapis służy do zapisania danych wejściowych, drugi -- do zapisania
wyniku danego zadania sporządzonego przez zadany algorytm.

\section{Algorytm dokładny}
\subsection{Przygotowanie danych}
Przed rozpoczęciem działania algorytmu wczytujemy klocki z pliku i tworzymy listę \textit{BrickGroupList} obiektów
\textit{BrickGroup}.Obiekt \textit{BrickGroup} zawiera tablicę ze wszystkimi 24 możliwymi orientacjami
klocka, indeks ostatnio rozważanego klocka zainiciowany wartością $-1$ oraz zmienne do zapamiętania przesunięcia.
Lista ruchów jest pusta. Dodatkowo wyszukujemy klocek o największej długości. Odbywa się to poprzez znalezienie
dla każdego klocka sześcianu o największej dowolnej współrzędnej i wybraniu z wyznaczonych wartości największej.
\subsection{Algorytm}
\subsubsection{Generowanie sekwencji ruchów}
Generujemy wszystkie możliwe permutacje listy klocków. Każdej permutacji odpowiada jedna lista \textit{BrickGroupList},
przy pomocy której generujemy sekwencję ruchów. Tworzymy iterator q wskazujący na pierwszy element listy. 
Sekwencja ruchów jest opisana przez fragment listy od początku do aktualnej pozycji iteratora q.
Iterator przesuwamy w następujący sposób:
\begin{itemize}
 \item Po przesunięciu iteratora do przodu (zainiciowanie iteratora również uważamy za przesunięcie iteratora do przodu)
generujemy pierwsze możliwe dopasowanie klocka w pierwszej orientacji, a następnie
wykonujemy sprawdzenie aktualnej sekwencji. Jeżeli możemy przesuwamy iterator dalej, w przeciwnym wypadku generujemy kolejne
dopasowanie dla klocka wskazywanego aktualnie przez iterator q i sprawdzamy sekwencję. Postępujemy tak aż sprawdzimy
wszystkie dopasowania dla wszystkich orientacji po czym cofamy iterator.
 \item Po cofnięciu iteratora generujemy kolejne dopasowanie klocka, sprawdzamy sekwencję i przesuwamy iterator do przodu.
 Jeżeli wygenerujemy wszystkie dopasowania dla danej orientacji przechodzimy do pierwszego dopasowania dla kolejnej orientacji.
 Jeżeli nie możemy wygenerować kolejnego dopasowania (osiągneliśmy ostatnie dopasowanie dla ostatniej orientacji), to 
 cofamy iterator. Jeżeli nie możemy cofnąć iteratora kończymy algorytm.
 \item Sprawdzenie sekwencji zwraca nam informację czy dana sekwencja opisuje bryłę o najdłuższej krawędzi większej niż
długość najdłuższego klocka. Jeżeli tak jest to w powyższych punkatch uznajemy, że nie możemy przesunąc iteratora dalej.
Jeżeli dostaniemy informację, że dana sekwencja opisuje prostopadłościan to zapamiętujemy ją.
 \end{itemize}
Wykonując powyższe ruchy znajdziemy wszystkie poprawne prostopadłościany. Za rozwiązanie przyjumujemy prostopadłościan o
największej objętości.
\subsubsection{Generowanie dopasowania klocka}
Możliwe dopasowania klocka generujemy próbując dołożyć klocek z trzech wzajemnie prostopadłych kierunków. Rozpatrujemy
tylko takie pozycje klocka, w których rzut  klocka na płaszczyznę prostopadłą do kierunku dokładania ma przynajmniej 
jeden kwadrat wspólny z rzutem bryły na tę samą płaszczyznę. Dopasowania generujemy w sposób systematyczny i uporządkowany.
\subsubsection{Sprawdzanie sekwencji}
Sprawdzenie sekwencji polega na weryfikacji, czy otrzymaliśmy poprawny prostopadłościan. Weryfikacji dokonujemy
poprzez znalezienie najmniejszego prostopadłościanu zawierającego bryłę i sprawdzeniu bryła go wypełnia.
 W wyniku sprawdzenia możemy otrzymać jedną z trzech informacji:
\begin{itemize}
 \item bryła jest prostopadłościanem
 \item bryła nie jest prostopadłościanem i nie spełnia warunków zadania
 \item bryla nie jest prostopadłościanem i spełnia warunki zadania.
\end{itemize}

\subsection{Złożoność}
Złożoność algorytmu generowania kolejnych permutacji wynosi $O(2^n)$.
Mamy drzewo T, którego korzeniem jest zbiór pusty. W węzłach drzewa znajdują się opisy ruchów.
Potomkami węzła są opisy ruchów wykonywanych przez powyższy algorytm dla układu klocków utworzonego 
przez sekwencję ruchów będącą ścieżką od korzenia do rozpatrywanego węzła. 
Złożoność operacji wykonywanych w każdym węźle wynosi $O(k^2n+k^3)$.
W przypadku pesymistycznym musimy przejrzeć całe drzewo, czyli złożoność pesymistyczna wynosi $O((k^2n+k^3)n^{n+1})$.
W przypadku średnim drzewo nie zawsze jest pełne co jednak nie zmniejsza złożoności. 


\section{Algorytmy aproksymacyjne}

\subsection{Algorytm klonowania}
Pierwszy z~algorytmów aproksymacyjnych jest autorstwa Pawła Baran.
W~niniejszym podrozdziale zostanie podany jego opis, pseudokod, koszty oraz
podsumowanie, w~którym przedstawione zostaną wady i~zalety algorytmu.

Algorytm składa się z~kilku kroków:
\begin{enumerate}
	\item grupowanie klocków,
	\item sprawdzenie kilku wybranych proporcji,
	\item wybranie optymalnego i~klonowanie.
\end{enumerate}

Algorytm stara się ułożyć prostopadłościan z~klocków, które mają swoje
kopie. Jeśli nie jest możliwe ułożenie z~klocków powtarzających się,
algorytm wykorzystuje wszystkie dostępne klocki, czyli w~najgorszym
przypadku zachowuje się jak algorytm dokładny, nie zmieniając kosztów
obliczeń w~stosunku do niego.

\subsubsection{Etap 1. Grupowanie klocków}
W~pierwszym etapie następuje stworzenie podzbiorów klocków zestawu, w
których następnie będą zamieszczane klocki z~identycznym kształtem.

Z~zestawu bierzemy kolejno klocki. Każdy wzięty klocek porównujemy
z~przedstawicielami różnych grup (z~każdej grupy po jednym). Dzięki
ustalonemu wymogowi poprawności zapisu klocka na wejściu (w~rozdziale
\ref{sec:zapis}) porównanie jest wygodne i~intuicyjne. Jeśli dany klocek
jest identyczny z~reprezentantem pewnej grupy, to jest zapisywany do niej.
Jeśli nie jest identyczny z~żadnym reprezentantem, to jest tworzona nowa
grupa i~on staje się jej reprezentantem (przedstawicielem). Etap się
kończy, gdy wszystkie klocki będą przypisane do grup. Najprostszą
i~najbardziej intuicyjną strukturą do zapisania takich powiązań będzie
tablica tablic identyfikatorów\footnote{tu: liczb całkowitych} klocków.
Złożoność obliczeniowa jest równa $O(n^2)$, gdzie $n$ jest proporcjonalna
do liczby klocków w~zestawie.

\subsubsection{Etap 2. Sprawdzenie kilku wybranych proporcji}
Mając pogrupowane klocki, dobieramy określone proporcje z~grup
najbardziej licznych. Proporcją nazywamy podzbiór zestawu klocków.
Pierwszą proporcją jest podzbiór, w~którym znajduje się po jednym
z~klocków z~grup zawierających co najmniej 2 elementy. Jeśli z~danego
,,podzestawu'' uda się ułożyć prostopadłościan, to jego rozwiązanie
wraz z~wynikiem jest zapisywane w~specjalnej tablicy. Następnie określana
jest kolejna proporcja. 1. sposób dobierania kolejnych proporcji przebiega
poprzez dodanie jednego elementu z~grupy, w~której jest największa
nadwyżka. Nadwyżkę definiujemy jako liczbę niewykorzystanych klocków danej
grupy przy klonowaniu. Chociaż klonowanie będzie dopiero w~następnym
kroku, to nie jest ono potrzebne do wyliczenia nadwyżki. Jeśli największa
nadwyżka występuje w~kilku grupach, to klocek jest wybierany
z~najliczniejszej grupy. Jeśli najliczniejszych grup jest kilka, to
z~pierwszej z~nich. 2. sposobem jest odrzucenie grupy najmniej licznej.
Jeśli jest ich kilka, to pierwszej z~nich. Po odrzuceniu następuje pobranie
z~każdej nieodrzuconej grupy po 1 elemencie, czyli zaczęcie na nowo, ale
bez klocków z~odrzuconej grupy.

2. sposób jest wybierany, gdy co najmniej
$90\%$ klocków z~najliczniejszej grupy nie jest wykorzystywana.
Liczba wykorzystanych klocków z~danej grupy jest zdefiniowana jako iloczyn
liczby klocków tej grupy z~proporcji i~liczby kopii proporcji (włącznie
z~daną proporcją). Kopią jest wykorzystanie klocków z~danych grup z~taką
samą proporcją jak wzorcowa przy wykorzystaniu klocków nieużytych. 

Wyjaśnione zostało pojęcie ,,wybrane'' w~nazwie etapu. Słowo ,,kilka''
należy interpretować jako liczbę 10. Czyli w~niniejszym etapie wybieramy
10 proporcji. Przedstawione sposoby mówią o~jak najmniej licznych
podzbiorach, z~których ułożenie prostopadłościanu jest najszybsze,
zakładając, że się da ułożyć z~nich prostopadłościan.

\subsubsection{Etap 3. Wybranie optymalnego i~klonowanie}
Ostatni etap jest jednocześnie najważniejszym punktem całego algorytmu.
Wyjaśnia sens działania algorytmu. Poprzez rozwiązania z~poprzedniego
punktu tworzymy kolejne prostopadłościany z~pozostałych klocków danych
grup. Nie musimy używać już algorytmu szukania prostopadłościanu, a~jedynie
kopiujemy ruchy przy powstawaniu wzorcowego prostopadłościanu. Te ,,klony''
są następnie dołączane do istniejącego prostopadłościanu.

W~niniejszym etapie wykorzystujemy proporcje z~10 wybranych, które dały
nam prostopadłościan. Może nie być ich w~ogóle i~wtedy algorytm nie
znajduje żadnego rozwiązania. Przy każdej proporcji dającej rozwiązanie
następuje klonowanie. Samą sekwencję ruchów przy danej proporcji kopiujemy
i~poprawiamy jego przesunięcie wzdłuż pewnej osi. Wybór osi w~naszym
przypadku jest istotny, ponieważ mamy ograniczenie górne najdłuższej
krawędzi prostopadłościanu, którym jest długość najdłuższej krawędzi
najdłuższego klocka. To ograniczenie narzuca sposób ,,dokładania''
kolejnych klonów do powstającego prostopadłościanu. Czyli dokładanie nie
może się odbywać wyłącznie poprzez dokładanie wzdłuż jednej osi. Jeśli
dostawiając kolejne klony nie przekraczamy ograniczenia, to samo dokładanie
może się odbyć wzdłuż jednej osi.

\subsubsection{Koszty działania algorytmu}
Algorytm działa bardzo dobrze tzn. jego koszt jest niski w~porównaniu
z~innymi algorytmami, gdy mamy wiele klocków ($n$), a~grup klocków
jest niewiele ($m$). Najlepsze warunki dla działania algorytmu są wtedy,
gdy $n$ jest nieporównywalnie większe od $m$ ($n>>m$). Prostopadłościany
układamy z~klocków, których liczność jest zależna od $m$, a~nie od $n$.
Gdy $m$ jest małe (w~stosunku do $n$), to w~algorytmie dokładnym, który
wykorzystujemy przy budowie prostopadłościanu dla danej proporcji,
zastępujemy $n$ liczbą $m$. Jest to zdecydowanie obniżenie kosztów
działania algorytmu. Ta część mieści się w~etapie 2. Koszt etapu 3 czyli
klonowanie jest mniejszy od budowy prostopadłościanu. Samo kopiowanie
sekwencji ruchów przebiega liniowo.

,,Krzywdzące'' jest ograniczenie najdłuższej krawędzi rozwiązania, toteż
algorytm nie będzie mógł być maksymalnie wykorzystany przy takich
warunkach. Również w~przypadkach, gdy $m$ jest porównywalne z~$n$,
algorytm traci swoje zastosowanie. Jednak podobne przypadki można znaleźć
dla innych algorytmów aproksymacyjnych.

\subsection{Algorytm aproksymacyjny II}
\subsubsection{Opis algorytmu}
Algorytm działa podobnie jak algorytm dokładny. Różni się tym, że nie rozpatruje za każdym razem wszystkich
możliwych dopasowań dla każdej orientacji tylko wybiera  $C$ $(C=const > 1)$ najlepszych par (orientacja, dopasowanie).
Tylko wybrane dopasowania są brane pod uwagę przy budowaniu sekwencji ruchów. Wybór następuje na podstawie funkcji 
oceny $f: B \rightarrow R$, gdzie B jest to zbiór brył. Dla danej bryły wartością funkcji $f(b)$ jest stosunek objętości
bryły $b$ (wyrażony w jednostkowych sześcianach) do objętości najmniejszego prostopadłościanu zawierającego bryłę $b$.
\subsubsection{Złożoność}
Z rozważań na temat złożoności algorytmu dokładnego i z ograniczenia budowy drzewa T przedstawionego w poprzednim paragrafie
wynika, że złożoność pesymistyczna tego algorytmu wynosi $O((k^2n+k^3)C^{n+1})$. Podobnie jak w przypadku algorytmu dokładnego
złożoność średnia jest taka sama jak pesymistyczna.

\section{Struktury danych}

\subsection{Zestaw klocków}
Zestaw klocków jest daną wejściową naszego programu. Niniejszy rozdział
dzieli się na dwie części.

\subsubsection{Format XML}\label{sec:zapis}
Zestaw klocków jest zapisywany w dokumencie XML, którego korzeniem jest
$bricks$. W korzeniu występują jedynie elementy $brick$. Każdy element
$brick$ jest opisem dokładnie jednego klocka. Każdy klocek należący do
zestawu i tylko taki klocek jest opisany za pomocą elementu $brick$.
Element $brick$ zawiera elementy $cube$. Każdy element $cube$ ma podane
wartości trzech atrybutów: $x$, $y$ i $z$.

Podobnie każdy klocek składa się
z sześcianów, które są niepodzielne. Sześcianem jest najmniejszy możliwy
klocek o długości krawędzi równej $1$. Z tego wynika, że krawędzie klocków
mają wartości całkowite równe co najmniej $1$. Każdy sześcian danego
klocka możemy zidentyfikować za pomocą trzech współrzędnych.

Klocek jest
umieszczany w układzie współrzędnych. Dzięki takiemu zabiegowi każde dwa
różne sześciany danego klocka mają różne wektory pomieszczenia. Wektorem
pomieszczenia sześcianu nazywamy wektor złożony z trzech współrzędnych
określających położenie sześcianu względem początku układu. Każdy
sześcian klocka i tylko taki sześcian jest opisany przez element $cube$
zawarty w $brick$ danego klocka.

Sposób na zapis danego klocka, a dokładnie -- jego sześcianów, jest ściśle
określony. Do metody zapisu potrzebna będzie definicja wagi zapisu klocka.
Wagą zapisu klocka nazywamy wartość:
\begin{equation}
	w(k) = \sum_{i=1}^n \rho (s_i)
\end{equation}
gdzie:
$k$ -- klocek,
$n$ -- liczba sześcianów klocka,
$s_i$ -- $i$--ty sześcian klocka,
$\rho (s)$ -- metryka euklidesowa wektora pomieszczenia sześcianu $s$.
Przyda się również definicja wagi zapisu klocka względem współrzędnej
$X_i$:
\begin{equation}
	w_i(k) = \sum_{j=1}^n s_j^i
\end{equation}
gdzie:
$k$ -- klocek,
$n$ -- liczba sześcianów klocka,
$s_j^i$ -- $i$--ta współrzędna $j$--tego sześcianu klocka.

Zapis klocka musi spełniać następujące warunki:
\begin{enumerate}
	\item każdy sześcian klocka ma współrzędne całkowite nieujemne,
	\item nie istnieje inny zapis klocka spełniającego warunek 1, którego
		waga jest mniejsza,
	\item waga zapisu względem współrzędnej $X_3=Z$ jest nie mniejsza niż 
		względem współrzędnej $X_2=Y$,
	\item waga zapisu względem współrzędnej $X_2=Z$ jest nie mniejsza niż 
		względem współrzędnej $X_1=Y$,
	\item sześciany mają ustalony porządek:
		\begin{enumerate}
			\item sześcian występujący przed innym w porządku ma nie wyższą
				pierwszą współrzędną,
			\item sześcian występujący przed innym o tej samej wartości
				pierwszej współrzędnej ma nie wyższą drugą współrzędną,
			\item sześcian występujący przed innym o tych samych wartościach
				pierwszej i drugiej współrzędnej ma niższą trzecią współrzędną.
		\end{enumerate}
\end{enumerate}
Ten zapis \textbf{gwarantuje}, że sprawdzenie, czy dwa klocki mają
identyczny kształt, sprowadza się do porównania współrzędnych kolejnych
sześcianów obu klocków. Jest to najszybsza i najwygodniejsza metoda
porównania, czy klocki są identyczne. Przez identyczne należy rozumieć,
że w rozwiązaniu klocek może zostać zastąpiony przez identyczny.

\subsubsection{Struktura danych}
Dane z dokumentu XML, czyli zestaw klocków, będą przechowywane w programie
w postaci listy obiektów klasy $Brick$. Klasa $Brick$ będzie zawierała
listę obiektów klasy $Cube$ odpowiadających sześcianom klocka z dokumentu
XML z zachowaniem kolejności\footnote{patrz: koniec poprzedniego
podrozdziału}.

Kolejność klocków z dokumentu XML będzie zachowana w liście obiektów klasy
$Brick$. Przy sekwencji ruchów będzie potrzebna identyfikacja klocków.
Klockom przypisze się jako identyfikator numer na liście zaczynając od $1$.
Jednak ten identyfikator nie jest potrzebny jako osobne pole w klasie
$Brick$.

\subsection{Sekwencja ruchów}
Sama sekwencja ruchów jest opisana za pomocą ciągu elementów. Elementem
nazywamy:
\begin{equation}
	e_i = (a_i,v,d,o)
\end{equation}
gdzie
$e_i$ -- $i$--ty element w ciągu,
$a_i$ -- indeks klocka użytego jako $i$--ty do konstrukcji sześcianu,
$d$ -- kierunek dokładania klocka,
$v$ -- wektor przemieszczenia klocka,
$o$ -- macierz obrotu.
Kierunkiem dokładania klocka może być $X$, $Y$ lub $Z$. Kierunek dokładania
jest związany z osią układu współrzędnych w przestrzeni 3D. Dokładany
klocek jest dokładany wzdłuż dodatniej półosi danej osi. Klocek zostaje
zatrzymany w dwóch przypadkach, gdy:
\begin{itemize}
	\item dochodzi do zetknięcia z powierzchnią innego klocka i kolejne
		przesunięcie o $1$ wzdłuż osi spowoduje najście na siebie co najmniej 
		jednego sześcianu z powstającego prostopadłościanu i co najmniej
		jednego sześcianu z dokładanego klocka,
	\item dochodzi do zetknięcia z płaszczyzną dwóch pozostałych osi.
\end{itemize}
Punkt, od którego zaczynamy przesuwanie klocka w kierunku powstającego
prostopadłościanu, nie jest istotny. Ważne jest jedynie przesunięcie
klocka względem dwóch pozostałych osi. Takim przesunięciem jest wektor
przemieszczenia, którego pierwsza współrzędna odpowiada przesunięciu
względem osi występującej wcześniej w zadanej kolejności osi układu.
Pozostała, druga współrzędna odpowiada przesunięciu względem pozostałej
osi. Mając podany kierunek tj. wzdłuż której osi dokładamy, wiemy,
jakim osiom odpowiadają dwie współrzędne wektora przemieszczenia klocka.

Przy macierzy obrotu należy pamiętać, że obrót wokół osi może wynosić
0 stopni, 90 stopni bądź wielokrotność 90 stopni.

\subsubsection{Struktura sekwencji ruchu}
Podaną wyżej definicję elementu przenosimy do klasy $Move$. Sekwencja
ruchów jest listą obiektów klasy $Move$. Wszelkie składowe elementu
odwzorujemy jako składowe klasy $Move$.

\subsubsection{Format XML}
Sekwencja ruchów jest zapisywana w dokumencie XML z korzeniem $moves$.
Obiekt klasy $Move$ z listy jest zapisywany do elementu $move$. W korzeniu
są jedynie zawarte te elementy $move$. Element $move$ zawiera po jednym
elemencie:
\begin{itemize}
	\item $brick_id$ -- identyfikator klocka,
	\item $vector$ -- wektor przemieszczenia klocka, gdzie jego współrzędne
		są kolejno opisane przez atrybuty $x1$ i $x2$,
	\item $d$ -- kierunek dokładania klocka, gdzie jedynym atrybutem
		jest wartość ze zbioru $P = {X,Y,Z}$,
	\item $o$ -- macierz obrotu.
\end{itemize}

\end{document}
