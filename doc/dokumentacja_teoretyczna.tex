\documentclass[12pt]{article}

\usepackage{polski}
\usepackage[utf8]{inputenc}
\usepackage[pdfborder={0 0 0}]{hyperref}
\usepackage{hyphenat}

\hypersetup{
	pdfinfo={
		Title = {Dokumentacja teoretyczna},
		Subject = {Dokumentacja teoretyczna},
		Author = {Paweł Baran, Andrzej Legucki},
		Keywords = {kostka, dokumentacja, teoretyczna}
	}
}

\title{
%\Huge{
%Teoria Algorytmów i Obliczeń
%}
\Huge{
%Laboratoria
Zima 2010 - kostka
} \\[0.5em]
\LARGE{
Dokumentacja teoretyczna
}
}

\author{
	Paweł Baran \and Andrzej Legucki
}

\begin{document}

\maketitle

\newpage

\tableofcontents

\newpage
\section{Wstęp}
Niniejsza dokumentacja powstała w ramach pierwszego etapu laboratoriów
z~przedmiotu Teoria Algorytmów i Obliczeń. W niniejszym dokumencie
przedstawiamy opis algorytmu dokładnego oraz opisy dwu algorytmów
aproksymujących do zadania. Problemem jest ułożenie największego
pełnego prostopadłościanu z klocków. Pojedynczym zadaniem problemu jest
ułożenie pełnego prostopadłościanu z zadanego zestawu klocków.
Rozwiązaniem zadania jest sekwencja ruchów klocków należących do zestawu
z zadania.

Oprócz algorytmów przedstawione zostały opisy struktur danych potrzebnych
do funkcjonowania podanych algorytmów. Opis struktur danych ułatwi ich
obsługę w zakresie zapisu zestawu klocków i zapisu sekwencji ruchów.
Pierwszy zapis służy do zapisania danych wejściowych, drugi -- do zapisania
wyniku danego zadania sporządzonego przez zadany algorytm.

\section{Algorytm dokładny}

\section{Algorytmy aproksymacyjne}

\subsection{Algorytm klonowania}

\subsection{Algorytm Andrzeja}

\section{Struktury danych}

\subsection{Zestaw klocków}

\subsection{Sekwencja ruchów}
\end{document}
